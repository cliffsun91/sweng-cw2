%% LyX 2.0.2 created this file.  For more info, see http://www.lyx.org/.
%% Do not edit unless you really know what you are doing.
\documentclass[english]{article}
\usepackage[T1]{fontenc}
\usepackage[latin9]{inputenc}

\makeatletter
%%%%%%%%%%%%%%%%%%%%%%%%%%%%%% User specified LaTeX commands.
\usepackage{babel}


\makeatother

\usepackage{babel}
\begin{document}

\title{Software Engineering CW2}


\author{Abhijit Chandgadkar, Osama Javed, }


\author{Joseph Slade, Cliff Sun, Luke Tomlin}


\date{27/11/12}

\maketitle

\section*{Process}

Acting as a small team tackling a completely unfamiliar codebase,
we took several steps towards implementing the desired goal, keeping
in mind extra aims to improve the existing code, make it maintainable
and make it extendable. 


\subsection*{Adding unit tests for existing functionality}


\paragraph*{What}

Whilst we had many ideas for improving the code itself, the most important
part was to make sure that we did not affect the existing functionality
beyond what we were asked to change. To do this, we began adding unit
tests that tested the existing functionality, so that we could easily
verify that our changes were not breaking anything. This was the first
step to making the code maintainable, much like we did in the first
lecture tutorial (Legacy Code). We decided to use $JUnit4$ to do
our unit testing.


\paragraph*{How}

We had aspirations to restructure the project to make it more easily
maintainable - this meant that we had to add tests for all of the
areas that we were planning to touch. As an example, the project specification
had us editing the $BillingSystem$ class. Upon inspection, we decided
what responsibilities this class had:
\begin{itemize}
\item Registers calls starting and ending between numbers
\item Processes these calls and applies them as bills to the corresponding
customers from the customer database

\begin{itemize}
\item In doing so, it calculates the tariff rates and uses some algorithm
(which we were changing in the specification) to decide the final
cost of the call.
\end{itemize}
\end{itemize}
Judging from these responsibilities, we added a variety of unit tests
covering basic and edge cases for each responsibility. For instance,
for $callInitiated$ and $callCompleted$, we added tests to make
sure that the specified events were added to the log. Additionally,
we fabricated some customer calls, with predefined times, and tested
these making sure that the resulting bill was equivalent to what we
expected. These calls were designed to cover all of the edge cases
as well, such as calls over peak time to off peak time and vice versa,
and calls of zero length, as well as extremely long calls. 


\subsection*{Forming specifications for new functionality}

In our project with Stephen, our entire project goals have been specified
by mouth only! As we saw in week 5 of the lecture (Spec by Example),
this is quite risky - we might completely misinterpret what Stephen
meant, and end up wasting time in the process. To remedy this, we
decided to create some Specifications, using the Fit framework.


\paragraph*{How}

We emulated a couple of customer calls, much like we did in the unit
tests, except as an HTML document using Fit. These showed when the
calls were made, and how long for, and what the resulting bill looked
like. (Of course we couldn't actually show these to Stephen, being
fictional after all. We assumed that these reflected the functionality
defined in the specification).


\subsection*{Adding unit tests for new functionality}

Before we could implement the required changes, we wanted to add unit
tests to cover what the expected outcome was going to be, an example
of Test-Driven Development. This meant that we could repeatedly test
the code as we implemented it to ensure that the desired goal was
being fulfilled. 


\paragraph*{How}


\subsection*{Implementing new functionality}


\paragraph*{What}


\paragraph*{How}


\subsection*{Refactoring code to improve maintainability and code health}

The existing code had multiple problems that we aimed to fix over
the course of the project. These were not all related to the goals
set out by Stephen, but as we would be looking after the codebase
for the forseeable future we decided that it would be wise to improve
the general health of the codebase to reduce headaches later. Some
examples:
\begin{itemize}
\item The $DaytimePeakPeriod$ class, which handled deciding whether calls
were onpeak or offpeak, was very rigidly implemented. The times for
the boundary between offpeak and peak were defined as ``magic numbers''
inside the code (in this case as 7 and 19, representing 7am and 7pm
respectively). This left no room for providing flexible peak and offpeak
times without directly editing the class itself. Additionally, if
different tariffs could have different peak times, perhaps changing
more than once throughout the day, it would require a whole overhaul
of the class. 
\item The $CallerStart$ and $CallerEnd$ classes, used $System.currentTimeMillis$
to get the time of call and end. This made testing difficult - to
construct a call using this implementation would require the test
to $sleep$ inbetween calling and ending the call, which would make
the testing process very slow indeed! 
\end{itemize}

\paragraph*{How}


\section*{Considerations}


\subsection*{Source control}

When we first received the code from Stephen, it was simply emailed
as a tarball. Keeping the code like this was definitely sub-optimal,
so we decided to host it in some form of source control (in our project
we used GitHub for ease of use - but in the context of Acme Telecom
we would most likely have used a private version!). This made collaboration
between the team members much easier, as well as keeping a timeline
of changes made to allow rollbacks.


\subsection*{Architecture changes}


\subsection*{Deployment}
\end{document}
